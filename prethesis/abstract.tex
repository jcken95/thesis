\begin{abstract}
Computationally expensive computer models, known as simulators, are fundamental to the modern scientific process. In recent years, there has been an increased interest in \emph{stochastic simulators} as analysts aim to model uncertainty in their predictions. Stochastic simulators are found in all areas of science. We focus on a particular simulator, known as the Athena simulator, which is used to assess the performance of offshore wind farms in their early life.

Athena is typical of many stochastic simulators in that it can take a long time to run and the variance exhibited by the simulator may depend on the simulator input. This poses multiple computational challenges for subjective Bayesians who wish to perform computer-intensive tasks such as sensitivity analysis or decision support. A common solution to the large computational cost of such analyses is to construct a fast statistical surrogate model, known as an \emph{emulator}, to produce an efficient approximation to the simulator with an appropriate quantification of uncertainty. Two popular emulators for stochastic simulators, HomGP, the homoscedastic Gaussian process, and HetGP, the heteroscedastic Gaussian process.

The first contribution of this thesis is the proposition of a method, based upon \citet{Kennedy2000}, which seeks to exploit fast approximations to stochastic simulators. We apply the general methodology to the Athena model in particular to construct an improved emulator. The approach considers an autoregressive structure for different `levels' of a simulator. We name the new approach stochastic multilevel emulation. \correction{The second contribution of this thesis is investigating algorithms to fit HetGP emulators. Many Bayesian emulators require computationally intensive methods such as Markov chain Monte Carlo to quantify uncertainty about emulator parameters. We recommend using a conjugate prior for HetGP which provides further computational savings and allows for tractable uncertainty quantification about emulator parameters}. Our Bayesian HetGPs are used to perform a sensitivity analysis as preparation for expert elicitation for parameters in the Athena simulator. The final contribution of this thesis is combining two distinct approaches. We show, via Bayesian Optimisation and History Matching, that decision making and decision support approaches can work in harmony. We apply this novel approach to construct a set of feasible solutions to a resource allocation problem within an offshore wind farm.
\end{abstract}
