
\begin{center}
\resizebox{0.4\textwidth}{!}{
\begin{tikzpicture}[FlowChart,                   % used are styles from tikzset FlowChart
    node distance = 5mm and 7mm,
      start chain = A going below                    % The nodes in the chain
                                                     % will be named by A-1, A-2, ...
                        ]
\node   [process] {Identify uncertain model parameters};               % A-1
\node   [process]   {Identify Experts}; %A-2
\node   [process]   {Perform initial elicitation. Identify a simple, but appropriate, distribution for each parameter.};%A-3
\node   [process]   {Construct emulator.};%A-4
\node   [decision] {Adequate emulator?};%A-5
\node   [process]   {Use sensitivity analysis to identify key inputs.};%A-6
\node   [process]   {Revise the elicitation. Give most attention to the most important inputs.};%A-7
\node   [process]   {Reconstruct emulator in light of refined uncertainty.};%A-8
\node   [decision] {Adequate emulator?};%A-9
\node   [process]   {Compute quanitites of interest.};%A-10
% lines not considered by join macro
\draw [arrow] (A-5.west) to ["no"] ++ (-1.5,0) |- (A-4);
\path   (A-5) to ["yes"] (A-6);
\draw [arrow] (A-9.west) to ["no"] ++ (-1.5,0) |- (A-8);
\path   (A-9) to ["yes"] (A-10);
    \end{tikzpicture}
}
\end{center}
